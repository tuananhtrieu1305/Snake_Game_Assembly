\subsubsection{\textit{Giải thích nội dung các thanh ghi, trên cơ sở đó giải thích cơ chế quản lý bộ nhớ của hệ thống trong trường hợp cụ thể.}}

\vspace{0.4cm}

\noindent\large 
Khi chương trình bắt đầu chạy, hệ điều hành tự động khởi tạo các thanh ghi, vùng nhớ và cấp phát không gian địa chỉ cho chương trình.

\vspace{0.4cm}

\noindent\large 
Tương ứng với các câu lệnh trong mã nguồn, nội dung các thanh ghi có thể thay đổi hoặc không.

\vspace{0.5cm}

\begin{itemize}
    \item \textbf{IP (Instruction Pointer):}
    \begin{itemize}
        \item \textbf{IP} là thanh ghi trỏ đến địa chỉ của lệnh tiếp theo sẽ được thực thi trong mã máy.
        \item Khi một chương trình được thực thi, \textbf{IP} được cập nhật để trỏ đến lệnh tiếp theo trong mã nguồn, giúp CPU biết lệnh nào sẽ được thực thi tiếp theo.
    \end{itemize}

    \item \textbf{DS (Data Segment) và SI (Source Index):}
    \begin{itemize}
        \item \textbf{DS} là thanh ghi chỉ đến phân đoạn dữ liệu, nơi dữ liệu chương trình được lưu trữ.
        \item \textbf{SI} thường được sử dụng trong các phép toán dữ liệu và là một trong các thanh ghi chỉ địa chỉ.
        \item Khi chương trình yêu cầu truy cập dữ liệu từ bộ nhớ, \textbf{DS} và \textbf{SI} thường được sử dụng để xác định vị trí của dữ liệu.
    \end{itemize}

    \item \textbf{SS (Stack Segment) và BP (Base Pointer):}
    \begin{itemize}
        \item \textbf{SS} là thanh ghi chỉ đến phân đoạn ngăn xếp (stack segment), nơi lưu trữ các giá trị cục bộ và địa chỉ của các hàm.
        \item \textbf{BP} thường được sử dụng để trỏ đến địa chỉ cơ sở của ngăn xếp (base of stack).
        \item Ngăn xếp được sử dụng để lưu trữ giá trị trung gian và địa chỉ trả về từ các hàm con trong quá trình thực thi chương trình.
    \end{itemize}

    \item \textbf{SP (Stack Pointer):}
    \begin{itemize}
        \item \textbf{SP} là thanh ghi chỉ đến đỉnh của ngăn xếp (top of stack).
        \item Khi dữ liệu được đẩy (push) hoặc rút (pop) ra khỏi ngăn xếp, \textbf{SP} sẽ thay đổi để chỉ đến vị trí mới nhất trong ngăn xếp.
        \item \textbf{SP} cũng được sử dụng để cấp phát không gian mới cho dữ liệu trong ngăn xếp.
    \end{itemize}
\end{itemize}

\vspace{0.6cm}

\noindent\large 
Cụ thể, trong trường hợp chương trình Assembly cho phép nhập vào một số và in ra màn hình giai thừa của số đó, cơ chế quản lý bộ nhớ như sau:

\vspace{0.5cm}

\begin{itemize}
    \item \textbf{Khởi tạo bộ nhớ:}
    \begin{itemize}
        \item Khi chương trình bắt đầu, hệ điều hành cấp phát các đoạn mã (code), dữ liệu (data) và ngăn xếp (stack) cho chương trình.
        \item Đoạn mã chứa các lệnh chương trình, đoạn dữ liệu chứa các biến, và ngăn xếp dùng để lưu trữ dữ liệu tạm thời.
    \end{itemize}

    \item \textbf{Phân đoạn bộ nhớ:}
    \begin{itemize}
        \item \textbf{CS} (Code Segment) trỏ tới đoạn mã của chương trình.
        \item \textbf{DS} (Data Segment) trỏ tới đoạn dữ liệu của chương trình.
        \item \textbf{SS} (Stack Segment) trỏ tới đoạn ngăn xếp của chương trình.
    \end{itemize}

    \item \textbf{Quản lý ngăn xếp:}
    \begin{itemize}
        \item Ngăn xếp được sử dụng để lưu trữ tạm thời dữ liệu và địa chỉ trả về khi gọi hàm.
        \item \textbf{SP} được khởi tạo với giá trị từ khai báo \texttt{.Stack 100}, nghĩa là kích thước ngăn xếp là 100 byte.
    \end{itemize}
\end{itemize}

\vspace{0.6cm}

\noindent\large 
Diễn giải nội dung của các câu lệnh trong mã nguồn và ảnh hưởng đến các thanh ghi:

\vspace{0.5cm}

\begin{itemize}
    \item \textbf{Khởi tạo đoạn dữ liệu:}
    \begin{lstlisting}[style=asm]
    mov ax, @data
    mov ds, ax  
    \end{lstlisting}
    \textbf{DS} trỏ tới đoạn dữ liệu, \textbf{AX} làm trung gian để chuyển địa chỉ đoạn dữ liệu vào \textbf{DS}.

    \item \textbf{Hiển thị thông báo:}
    \begin{lstlisting}[style=asm]
    mov ah, 9              
    lea dx, msg1
    int 21h
    \end{lstlisting}
    hoặc
    \begin{lstlisting}[style=asm]
    mov ah, 9              
    lea dx, msg2
    int 21h
    \end{lstlisting}
    Gắn \textbf{AH} = 9, \textbf{DX} trỏ tới địa chỉ chuỗi thông báo cần in ra màn hình.

    \item \textbf{Nhập dữ liệu:}
    Gọi hàm nhập số:
    \begin{lstlisting}[style=asm]
    call NhapSo 
    \end{lstlisting}
    Trong hàm \texttt{NhapSo}:
    \begin{lstlisting}[style=asm]
    NhapSo proc                
    mov x, 0               
    mov y, 0               
    mov bx, 10             
    nhap:   
        mov ah, 1         
        int 21h 
        cmp al, 13         
        je nhapxong
        sub al, '0'        
        mov ah, 0
        mov y, ax         
        mov ax, x
        mul bx            
        add ax, y       
        mov x, ax          
        jmp nhap        
    nhapxong:
        mov ax, x          
    ret
    NhapSo endp 
    \end{lstlisting}
    \begin{itemize}
        \item Gắn \textbf{AH} = 1 để nhập ký tự từ bàn phím, ký tự nhập vào lưu trong \textbf{AL}.
        \item Sau khi nhập xong, số nhập được gán vào thanh ghi \textbf{AX}.
    \end{itemize}

    \item \textbf{Tính giai thừa:}
    \begin{lstlisting}[style=asm]
    inc n                  
    mov bx, 1              
    mov ax, 1     
    Cal:
        cmp bx, n          
        je break           
        mul bx            
        inc bx           
        jmp Cal     
    \end{lstlisting}
    \textbf{AX} lưu kết quả phép nhân, \textbf{BX} lưu biến đếm sử dụng trong nhãn \texttt{Cal}.

    \item \textbf{In ra màn hình:}
    Gọi hàm in số:
    \begin{lstlisting}[style=asm]
    call InSo 
    \end{lstlisting}
    Trong hàm \texttt{InSo}:
    \begin{lstlisting}[style=asm]
    InSo proc                 
    push ax
    push bx
    push cx
    push dx
    
    mov bx, 10             
    mov cx, 0
    beforePrint:
        mov dx, 0
        div bx            
        push dx           
        inc cx          
        cmp ax, 0          
        jg beforePrint
    print:
        pop dx      
        mov ah, 2          
        add dx, '0'      
        int 21h      
        loop print       
    
    pop dx
    pop cx
    pop bx
    pop ax
    ret
    InSo endp
    \end{lstlisting}
    \begin{itemize}
        \item Đẩy \textbf{AX}, \textbf{BX}, \textbf{CX}, \textbf{DX} vào ngăn xếp để bảo vệ dữ liệu.
        \item Phân tách từng chữ số từ \textbf{AX} đưa vào stack, rồi lần lượt lấy ra in ra màn hình.
    \end{itemize}
\end{itemize}